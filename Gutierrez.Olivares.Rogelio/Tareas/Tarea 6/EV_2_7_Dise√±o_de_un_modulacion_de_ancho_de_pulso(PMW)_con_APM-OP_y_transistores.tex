\documentclass[12pt,a4paper]{report}
\usepackage[latin1]{inputenc}
\usepackage[spanish]{babel}
\usepackage{amsmath}
\usepackage{amsfonts}
\usepackage{amssymb}
\usepackage{graphicx}
\usepackage[left=2cm,right=2cm,top=2cm,bottom=2cm]{geometry}
\author{Gutierrez Olivares Rogelio}
\title{PWM}
\begin{document}
\maketitle
\section{Que es}
Se usa para controlar el ancho de una senal digital con el proposito de controlar a su vez de  la potencia que se entrea a ciertos dispositivos. Modificando el ancho del pulso activo (que esta en On) se controla la cantidad de corriente que fluye hacia el dispositivo.
\section{Funcionamiento}
Un PWM funciona como un interruptor, que constantemente se activa y desactiva, reulando la cantidad de corriente y por ende potencia, que se entrega al dispositivo que se desea controlar. Estos dispositivos pueden ser montores CC o fuentes de luz en CC, entre otros.\\
Si un motor alimentado con 12 voltios, recibe todo el tiempo la corriente que este pide y entrega la maxima potencia, si es alimentado con 0 voltios, no recibe corriente y no obtiene potencia.\\
En un sistema PWM el motor recibe corriente por un tiempo y deja de recibir por otro, repitiendose ente proceso continuamente. Si se aumenta el tiempo en que el pulso esta en nivel potencia. 
\section{Ventajas}
La principal ventaja es la eficiencia energetica. El circuito que tenga este metodo de control, entrega a la cargga una cantidad de potencia que es proporcional a la potencia que necesita para realizar su trabajo.
\\
1-Si se necesita aumentar la velocidad de un motor se incrementa la potencia que se le entrega (ciclo de trabajo mayor)
\\
2-Si se necesita disminuir la velocidad de un motor se disminuye la potencia que se entrea. (ciclo de trabajo menor)
\section{Aplicacion}
Se utiliza reularmente para:
\\
1-Controles de velocidad variables para motores CC
\\
2-Dimmers para sistemas de iluminacion con LEDs
\end{document}