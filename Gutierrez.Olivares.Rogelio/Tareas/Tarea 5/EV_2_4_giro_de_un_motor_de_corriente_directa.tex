\documentclass[12pt,a4paper]{report}
\usepackage[latin1]{inputenc}
\usepackage[spanish]{babel}
\usepackage{amsmath}
\usepackage{amsfonts}
\usepackage{amssymb}
\usepackage{graphicx}
\usepackage[left=2cm,right=2cm,top=2cm,bottom=2cm]{geometry}
\author{Gutierrez Olivares Rogelio}
\title{Giro de motor de corriente directa}
\begin{document}
\maketitle
\section{Que es}
Es una maquina que convierte energia electrica en mecanica, provocando un movimiento rotatorio, ggracias a la accion de un campo magnetico.
Un motor de corriente continua se compone, principamente, de dos partes: El estator de soporte mecanico al aparato y contiene los polos de la maquina, que pueden ser o bien devanados de hilo de cobre sobre un nucleo de hierro, o imanes permanentes. El rotor es generalemente de forma cilindrica, tambien devanado y con nucleo, alimentado con corriente directa a traves las delgas, que estan en contacto alternamente con escobillas fijas.
\section{Principios de funcionamiento}
Una espira de material conductor inmersa en un campo magnetico, a la cual se le aplica una diferencia de potencial entre sus extremos, de forma que a traves de la misma circula una corriente.\\
Para este caso la espira constituye el rotor del motor, y los imanes que producen el campo maggnetico constituye el estator.
Entonces, dado que cuando un conductor, por el que pasa una corriente electrica, se encuentra inmerso en un campo manetico, este experimenta una fuerza degun la Ley de Lorentz. Donde dicha fuerza, denominada fuerza de Lorentz, es perpendicular al plano formado por el campo manetico y la corriente, y su magnitud esta dada por:\\
F=B*L*I*sen(0)
\section{Fuerza contraelectromotriz}
Es la tension que se crea en los conductores de un motor como consecuencia del corte de las lineas de fuerza. La polaridad de la tension en los generadores es opuesto a la aplicada en los bornes del motor. Durante el arranque de un motor de corriente continua se produce fuertes picos de corriente ya que, al estar la maquina parada, no hay fuerza contraelectromotriz y el bobinado se comporta como un simple conductor de baja resistencias. La fuerza contraelectromotriz en el motor depende directamentede la velocidad de giro del motor y del flujo maggnetico del sistema inductor.
\section{Sentido de giro}
En maquinas de corriente directa de mediana y ran potencia, es comun la fabricacion de rotores con laminas de acero electrico para disminuir las perdidas asociadas a los campos magneticos variables, como las corrientes de Foucault y las producidas por isteresis.
\section{Reversibilidad}
Por reversibilidad entre el motor y el generador se entiende que si se hace girar el rotor, se produce en el devanado inducido una fuerza electromotriz capaz de tranformarse en energia electrica. En cambio, si se aplica tension continua al devanado inducido del generador a traves del colector delga, el comportamiento de la maquina ahora es de motor, capas de transformar la fuerza contraelectromotriz en eneria mecanica. En ambos casos el inducido esta sometido a la accion del campo magnetico del inductor principal en el estator.

\end{document}