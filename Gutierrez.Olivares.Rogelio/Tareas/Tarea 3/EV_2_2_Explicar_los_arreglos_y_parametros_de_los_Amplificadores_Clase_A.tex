\documentclass[12pt,a4paper]{report}
\usepackage[latin1]{inputenc}
\usepackage[spanish]{babel}
\usepackage{amsmath}
\usepackage{amsfonts}
\usepackage{amssymb}
\usepackage{graphicx}
\usepackage[left=2cm,right=2cm,top=2cm,bottom=2cm]{geometry}
\author{Gutierrez Olivares Rogelio}
\title{Arreglos y Parametros de los amplificadores clase A}
\begin{document}
\maketitle
\chapter{Que son los amplificadores clase A?}
Un amplificador de clase A es aquel que presenta a su salida una senal copia de la entrada, pero amplificada y sin distorsion.En este caso la maxima senal de salida se obtendra cuando el punto estatico coincida con el centro de la recta de carga, consiguiendo, por tanto, la maxima potencia de salida.\\
Se puede presentar dos casos en cuanto a la conexion de la carga, esto es, que la carga, a la que hay que aplicar la potencia, sea externa al circuito o, que esta sea la propia carga del transistor. A pesar de su escasa utilizacion, cuando se emplea suele coincidir con esta ultima disposicion.
\chapter{Arreglos de amplificadores clase A}
Esta amplificacion presenta el incoveniente de generar una fuerte y constante emision de calor. No obstante, los transistores de salida estan siempre a una temperatura fija y sin alteraciones.\\
En ggeneral, se afirma que esta clase de amplificacion es frecuente en circuitos de audio y en los equipos domesticos de gama alta, ya que proporcionan una calidad de sonido potente y de muy buena calidad.\\
Los amplificador de clase A a menudo consisten en un transistor de salida conectado al positivo de la fuente de alimentacion y un transistor de corriente constante conectado de la salida al negativo de la fuente de alimentacion.\\
La senal del transistor de salida modula tanto el voltaje como la corriente de salida. Cuando no hay senal de entrada, la corriente de polarizacion contante fluye directamente del positivo de la fuente de alimentacion negativo, resultado que no hay corriente de salida, se gasta mucha corriente. Algunos amplificadores de clase A mas sofisticados tienen dos transistores de salida en configuracion push-pull.
\section{Ventajas}
La clase A se refiera a una etapa de salida con una corriente de polarizacion mayor que la maxima corriente de salida que dan, de tal forma que los transistores de salida siempre estan consumiendo corriente. La gran ventaja de la clase A es que es casi lineal, y en consecuencia la distorsion es menor.
\section{Desventajas}
La gran desventaja de la clase A es que es poco eficiente, se requiere un amplificador de clase A muy grande para dar 50 W, y ese amplificador usa mucha corriente y se pone muy alta temperatura.
\chapter{Parametros de los amplificadores Clase A}
Impedancia de entrada Zi: es la resistencia entre las entradas del amplificador.\\
Impedancia de salida Zo: es la resistencia que se observa a la salida del amplificador.\\
Ganancia en lazo abierto Aol: Indica la ganacia de tension en ausencia de realimentacion. Se puede expresar en unidades naturales.\\
Tension en modo comun Vcm: es el valor promedio de tension aplicado a ambas entradas de amplificador operacional.\\
Voltaje de desequilibrio "offset" de entrada V10: Es la diferencia de tension, entre las entradas de un amplificador.\\
Corriente de desequilibrio de entrada I10: es la diferencia de corriente entre las dos entradas del amplificador operacional, que hace que su salida tome el valor cero.\\
Voltaje de entrada diferencial Vid: es la mayor diferencia de tension entre las entradas del operacional que mantienen el dispositivo dentro de las espesificaciones.
\\
Corriente de polarizacion de entrada Iib: corriente media que circula por las enttradas del operacional en ausencia de senal.\\
\paragraph{Conclusion.}
Los amplificadores como el de Clase A son amplificadores operacionales que permieten aumentar la corriente dentro de los circuitos dando un gran parametro para diversar las funciones del mismo por lo que nos abre las puertas a nuevos proyectos controlando el flujo de corriente y voltaje danto un mejor sentido a nuestros trabajos y proyectos.


\end{document}