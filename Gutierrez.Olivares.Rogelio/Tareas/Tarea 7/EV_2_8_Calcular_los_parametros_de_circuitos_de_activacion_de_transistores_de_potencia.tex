\documentclass[12pt,a4paper]{report}
\usepackage[latin1]{inputenc}
\usepackage[spanish]{babel}
\usepackage{amsmath}
\usepackage{amsfonts}
\usepackage{amssymb}
\usepackage{graphicx}
\usepackage[left=2cm,right=2cm,top=2cm,bottom=2cm]{geometry}
\author{Gutierrez Olivares Rogelio}
\title{Calculo de Activacion de Transistor de potencia}
\begin{document}
\maketitle
\paragraph{Parametros}
Los parametros tomados en cuenta para los transistores son los siguientes:\\
1-Impedancia de entrada: Alta(1010ohmios)
\\
2-Ganancia de corriente: Alta (107)
\\
3-Resistencia ON (saturacion): Media/alta
\\
4-Resistencia OFF (corte): Alta
\\
5-Voltaje aplicable: Alto(1000V)
\\
6-Maxima temperatura de operacion: Alta (200grados) 
\\
7-Frecuencia de trabajo: Alta (100-500 Khz

\paragraph{Calculos}
\begin{center}
Tenemos que la tension del colector-emisor viene dada como:\\
Vce=Vcc-R*ic\\
Sustituyendo, tendremos que:
\\\
Vce=Vcc-R(Vcc/R)(t/tr)=Vcc*(1-(t/tr))
\\\
Nosotros asumiremos que el Vce en saturacion es despreciable en compracion con Vcc asi, la potencia instantanea por el transistor durante este intervalo viene dada por:\\\
p=Vce*Ic=Vcc*Icmax(t/tr)(1-(t/tr))\\\
La energia Wr, disipada en el transistor durante el tiempo de subida esta dada por la integral de la potencia durante el intervalo del tiempo de caida, con el resultado:\\
Wr=((vcc*Icmax)/4)((2tr)/3)
\\
De forma similar, la energia (Wf) disipada en el transistor durante el tiempo de caida, viene dado como:
\\
Wf=((vcc*Icmax)/4)((2tf)/3)\\
La potencia media resultante dependeria de la frecuencia con que se efectue la conmutacion :
\\
Pav= f*(Wr+Wf)\\
Un ultimo paso a considerar tr despreciable frente a tf, con lo que no cometeriamos un error apreciable si finalmente dejamos la potencia media, tras sustituir como:
\\
Pc(av)=((Vcc*Icmax)/6)(tf)(f)

\end{center}



\end{document}